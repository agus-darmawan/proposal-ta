\chapter*{ABSTRAK}
\begin{center}
  \large
  \textbf{PENGEMBANGAN ROBOT \emph{QUADRUPED-LEGGED} UNTUK ESTIMASI POSISI KOMPONEN \emph{OVERHEAT} PADA GARDU LISTRIK BERBASIS KAMERA TERMAL}
\end{center}
\addcontentsline{toc}{chapter}{ABSTRAK}
% Menyembunyikan nomor halaman
\thispagestyle{empty}

\begin{flushleft}
  \setlength{\tabcolsep}{0pt}
  \bfseries
  \begin{tabular}{ll@{\hspace{6pt}}l}
  Nama Mahasiswa / NRP&:& I Wayan Agus Darmawan 5024211070\\
  Departemen&:& Teknik Komputer FTEIC - ITS\\
  Dosen Pembimbing&:& 1. Prof.Dr.Ir. Mauridhi Hery Purnomo, M.Eng.\\
  & & 2. Muhtadin, S.T., M.Sc.\\
  \end{tabular}
  \vspace{4ex}
\end{flushleft}
\textbf{Abstrak}

% Isi Abstrak
Peningkatan konsumsi listrik di Indonesia memerlukan pengembangan infrastruktur kelistrikan, terutama pada gardu induk yang berperan penting dalam mengalirkan listrik kepada konsumen. Salah satu masalah yang sering terjadi pada transformator gardu adalah \emph{overheating}, yang berpotensi menimbulkan kerusakan serius dan gangguan distribusi listrik. Untuk mengatasi masalah ini, penelitian ini mengusulkan pengembangan sistem pemantauan overheat berbasis \emph{Autonomous Quadruped Robot} dan \emph{Thermal Camera}. Sistem ini dirancang untuk dapat beroperasi secara otonom di sekitar gardu listrik dan mendeteksi suhu abnormal pada transformator secara \emph{real-time}. Robot yang digunakan dilengkapi dengan kamera termal yang terintegrasi dengan model deteksi objek \emph{YOLOv8}, serta sistem navigasi berbasis \emph{Robot Operating System (ROS)}  yang memungkinkan pergerakan mandiri di lingkungan yang kompleks. Penelitian ini bertujuan untuk meningkatkan efisiensi dan keselamatan dalam pemantauan gardu listrik, serta meminimalkan risiko kesalahan manusia dan biaya operasional. Hasil penelitian diharapkan dapat memberikan kontribusi pada teknologi robotika di bidang pemantauan infrastruktur kelistrikan.

\vspace{2ex}
\noindent
\textbf{Kata Kunci:} \emph{Overheat}, Transformator,\emph{Autonomous Quadruped Robot, Thermal Camera, YOLOv8}