\chapter*{ABSTRAK}
\begin{center}
  \large
  \textbf{PENGEMBANGAN ROBOT \emph{QUADRUPED-LEGGED} UNTUK ESTIMASI POSISI KOMPONEN \emph{OVERHEAT} PADA GARDU LISTRIK BERBASIS KAMERA TERMAL}
\end{center}
\addcontentsline{toc}{chapter}{ABSTRAK}
% Menyembunyikan nomor halaman
\thispagestyle{empty}

\begin{flushleft}
  \setlength{\tabcolsep}{0pt}
  \bfseries
  \begin{tabular}{ll@{\hspace{6pt}}l}
  Nama Mahasiswa / NRP&:& I Wayan Agus Darmawan 5024211070\\
  Departemen&:& Teknik Komputer FTEIC - ITS\\
  Dosen Pembimbing&:& 1. Prof.Dr.Ir. Mauridhi Hery Purnomo, M.Eng.\\
  & & 2. Muhtadin, S.T., M.Sc.\\
  \end{tabular}
  \vspace{4ex}
\end{flushleft}
\textbf{Abstrak}
% Isi Abstrak
Peningkatan konsumsi listrik di Indonesia mendorong pengembangan infrastruktur kelistrikan, terutama gardu listrik yang rentan terhadap masalah \emph{overheating}. Masalah ini dapat menyebabkan kerusakan serius dan gangguan distribusi listrik. Penelitian ini mengusulkan sistem pemantauan \emph{overheat} berbasis \emph{autonomous quadruped legged robot} yang dilengkapi \emph{thermal camera} dan model deteksi objek \emph{YOLOv8}. Robot ini dirancang untuk mendeteksi suhu abnormal pada komponen gardu listrik secara \emph{real-time}, dengan sistem lokalisasi ganda yang menggabungkan GPS RTK dan \emph{Direct LiDAR-Inertial Odometry (DLIO)}. Data dari IMU dan odometri difilter menggunakan \emph{Extended Kalman Filter} (EKF) untuk mengurangi noise, sehingga menghasilkan estimasi posisi yang stabil sebelum digunakan dalam proses DLIO. Integrasi ini memungkinkan robot untuk mengikuti rute patroli dengan akurasi tinggi, menghindari hambatan menggunakan \emph{Braitenberg controller}, dan memastikan navigasi aman di lingkungan yang penuh tantangan. Sistem dilengkapi antarmuka \emph{web} untuk memudahkan operator dalam memantau dan mengontrol robot dari jarak jauh. Hasil penelitian ini diharapkan berkontribusi pada pengembangan teknologi robotika untuk pemantauan infrastruktur kelistrikan yang lebih efisien dan andal.

\vspace{2ex}
\noindent
\textbf{Kata Kunci:} \emph{Overheat}, \emph{Quadruped robot}, \emph{thermal camera}, YOLOv8, Direct LiDAR-Inertial Odometry
