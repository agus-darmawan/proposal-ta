\chapter*{ABSTRACT}
\begin{center}
  \large
  \textbf{MAPPING OF TRANSFORMER OVERHEATING IN ELECTRIC SUBSTATIONS BASED ON AUTONOMOUS QUADRUPED ROBOTS AND THERMAL CAMERAS}
\end{center}
% Menyembunyikan nomor halaman
\thispagestyle{empty}

\begin{flushleft}
  \setlength{\tabcolsep}{0pt}
  \bfseries
  \begin{tabular}{lc@{\hspace{6pt}}l}
  Student Name / NRP&: &I wayan Agus Darmawan / 5024211070\\
  Department&: &Aerospace Engineering FTD - ITS\\
  Advisor&: &1. Prof.Dr.Ir. Mauridhi Hery Purnomo, M.Eng.\\
  & & 2.  Muhtadin, S.T., M.Sc.\\
  \end{tabular}
  \vspace{4ex}
\end{flushleft}
\textbf{Abstract}

The increasing electricity consumption in Indonesia requires the development of adequate electrical infrastructure, especially in substations, which play a critical role in distributing electricity to consumers. One of the frequent issues in substations is transformer overheating, which poses serious risks of damage and disruption to electricity distribution. To address this problem, this research proposes the development of an overheating monitoring system based on an Autonomous Quadruped Robot and Thermal Camera. The system is designed to autonomously operate around substations and detect abnormal temperatures in transformers in real-time. The robot is equipped with a thermal camera integrated with the YOLOv8 object detection model and utilizes a navigation system based on the Robot Operating System (ROS), allowing independent movement in complex environments. This research aims to enhance efficiency and safety in substation monitoring while minimizing human error and operational costs. The expected outcome of this study is to contribute to robotics technology in the field of electrical infrastructure monitoring.

\vspace{2ex}
\textbf{Keywords:} Overheating, Transformer, Autonomous Quadruped Robot, Thermal Camera, YOLOv8