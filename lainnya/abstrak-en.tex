\chapter*{ABSTRACT}
\begin{center}
  \large
  \textbf{Development of a Quadruped-Legged Robot for Position Estimation of Overheat Components in Electrical Substations Using a Thermal Camera}
\end{center}
% Menyembunyikan nomor halaman
\thispagestyle{empty}

\begin{flushleft}
  \setlength{\tabcolsep}{0pt}
  \bfseries
  \begin{tabular}{lc@{\hspace{6pt}}l}
  Student Name / NRP&: &I Wayan Agus Darmawan / 5024211070\\
  Department&: &Commputer Engineering FTEIC - ITS\\
  Advisor&: &1. Prof.Dr.Ir. Mauridhi Hery Purnomo, M.Eng.\\
  & & 2.  Muhtadin, S.T., M.Sc.\\
  \end{tabular}
  \vspace{4ex}
\end{flushleft}
\textbf{Abstract}
% Abstract
The increasing electricity consumption in Indonesia drives the development of electrical infrastructure, particularly electrical substations, which are vulnerable to \emph{overheating} issues. These problems can cause severe damage and disrupt power distribution. This study proposes an \emph{overheat} monitoring system based on an \emph{autonomous quadruped legged robot} equipped with a \emph{thermal camera} and an object detection model using \emph{YOLOv8}. The robot is designed to detect abnormal temperatures in substation components in \emph{real-time}, with a dual localization system that combines GPS RTK and \emph{Direct LiDAR-Inertial Odometry (DLIO)}. Data from the IMU and odometry are filtered using the \emph{Extended Kalman Filter} (EKF) to reduce noise, providing a stable position estimate before being used in the DLIO process. This integration allows the robot to follow patrol routes with high accuracy, avoid obstacles using a \emph{Braitenberg controller}, and ensure safe navigation in challenging environments. The system is equipped with a \emph{web} interface to facilitate remote monitoring and control by the operator. The results of this research are expected to contribute to the development of more efficient and reliable robotic technologies for electrical infrastructure monitoring.

\vspace{2ex}
\noindent
\textbf{Keywords:} \emph{Overheat}, \emph{Quadruped robot}, \emph{thermal camera}, YOLOv8, Direct LiDAR-Inertial Odometry
